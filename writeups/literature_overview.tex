\documentclass{article}[10pt]      % use "amsart" instead of "article" for AMSLaTeX format
\usepackage{geometry}                       % See geometry.pdf to learn the layout options. There are lots.
\geometry{letterpaper}                          % ... or a4paper or a5paper or ... 
%\geometry{landscape}                       % Activate for rotated page geometry
%\usepackage[parfill]{parskip}          % Activate to begin paragraphs with an empty line rather than an indent
\usepackage{graphicx}               % Use pdf, png, jpg, or eps§ with pdflatex; use eps in DVI mode
                                % TeX will automatically convert eps --> pdf in pdflatex   
\usepackage{caption}
\usepackage{subcaption}
\captionsetup{justification=raggedright,singlelinecheck=false}

\usepackage{amsmath}
\usepackage{amsthm}
\usepackage{amsfonts}
\usepackage{verbatim}
\usepackage{bm}


\usepackage{tikz}
\usetikzlibrary{fit,positioning}

\newcommand{\cov}{\mathrm{cov}}

%\usepackage{minted}


\begin{document}
\title{Literature Survey}
\author{Luke Rast}
\maketitle


This is where I put together a summary of the other literature related to this problem.
There is a large amount of it to go through.
Hopefully this will help with sorting through all of the things that have been done previously.


\section{Individual Summaries}
Summaries of individual works written as I read through them in detail







\section{Questions / to dos}

\begin{enumerate}
  \item Look into importance-weighted ERM
\end{enumerate}

One critical thing that I need to demonstrate for this problem is the fidelity of the method for different tasks, and the behavior of the learned CGF under different conditions. 
This is probably the most important issue. Even more so than than coming up with different / novel methods for adjusting the networks. That said, a Bayesian adjustment based on the distribution is still appealing.


The core thing that I still haven't figured out is: What are my actual claims?



\section{Distribution Shift}

Given inputs $\bm{x}$ and outputs $y$, we can classify the different types of distribution shift that can occur.
`Covariate shift' leaves the distribution $p(y|\bm{x})$ unchanged, while $p(x)$ changes: the feature distribution may change, but they way that features generate labels is unchanged
`Label shift' leaves $p(\bm{x}|y)$ fixed while $p(y)$ changes: the prevalence of different labels may change, but the signatures of the labels are unchanged.
Meanwhile, `concept drift' leaves the one (or both) of the marginal distributions unchanged, while the conditional distribution changes.



\subsection{Detecting and Correcting for Label Shift with Black Box Predictors - 2018 \cite{lipton_detecting_2018}}
(This is a good baseline for label shift detection and adaptation methods. It is a simple method using empirical frequencies.)

Black box shift estimation: Focuses on the case of \textit{label shift} detection and adaptation.

\noindent \textbf{Detection:} First, use a method of moments involving empirical label frequencies and confusion matrices to estimate the observed label distribution.
Next, detect label shifts by finding changes in the distribution of predictions (via KS test).

\noindent \textbf{Correction:} Weighted empirical risk minimization with importance weights determined from the detection approach.
In other words, the loss function is weighted by the relative prevalence of a given label.

\noindent \textbf{Relationship to mine:}
This is one-dimensional, other work (On the Decreasing Power of Kernel and Distance based Nonparametric Hypothesis Tests in High Dimensions) points to an issue with high dimensional representaions, ours is a medium dimensional representation.
This work uses a method of moments approximation for the distribution of classifier outputs, while we use the internal statistic of neural network activity.
This is a simple starting point for explaination, and also a work that we are fairly directly expanding upon.


\subsection{A Unified View of Label Shift Estimation - 2020 \cite{garg_unified_2020}}
Again focuses on label shift, building on the previous work, among others. Again focuses on importance weight estimation.
Contrasts confusion matrix approaches with label shifts fit by fitting minimum KL estimates, assuming that the estimator produces \textit{calibrated} predictions of output values.
Importantly, this work generalizes the formalism to include feature vectors that are (random) functions of inputs.
It focuses on predictor outputs, but the approach should apply to our learned features as well.
It also shows us what we need in our probability estimators: calibration.


\subsection{Understanding new tasks through the lens of training data via exponential tilting - 2023 \cite{maity_understanding_2023}}
Uses exponential tilting to find importance weights for fine-tuning. Emphasizes that these importance weights are not limited to the label shift problem or the covariate shift problem. Both paramaeter and normalization values are fit by optimization of KL-divergence between the tilted marginal stimulus distribution and the test-time data.



\subsection{Covariate Shift Adaptation by Importance Weighted Cross Validation - 2007 \cite{sugiyama_covariate_2007}}
Early idea to use importance weights to correct: 1. loss function to get consistent estimators, 2. cross validation to get unbiased risk estimator.
Simple estimates of density, .



\section{Saddlepoint Approximation}






\section{Misc}
\subsection{Mini Summaries}
\cite{hendrycks_baseline_2017}: Uses neural network classifier output values to find individual examples that are in / out of distribution.
\cite{garg_leveraging_2022}: Similar idea for using black-box calssifier softmax outputs to predict accuracy, compares to netowrks calibrated on target data distributions. - good baselines for shift detection.


\cite{namkoong_minimax_2024}: defines a notion of the `stability' of a model under distribution shifts:
\begin{equation}
  I_y(P) = \min_Q \{D_{kl}(Q | P) | \mathbb{E}_Q[R] \ge y  \}
\end{equation}
here $R \sim P$ is the loss distribution of a model, and $y$ is a threshold value of loss.
So, the metric is to find the \textit{closest} distribution of outputs KL-wise that exceeds acceptable losses.
The goal here is to predict which models will fail on novel datasets by examining their performance distributions....
If $R$ is distributed according to $P$, then $I_y(P)$ is the large deviations rate function (as a function of $y$).

\cite{guggilam_anomaly_2021}: Large deviations approach to anomaly detection. Uses some sort of projection approach to compute the rate function. Uses threshold applied to the rate function as an anomaly detection approach.

\cite{rabanser_failing_2019}: tests of different methods for detecting shifts.



\section{To look at}
Previous works on exponential tilting
- finance
- ML


\section{Adaptation}




\section{Exponential tilting}






\bibliography{literature_overview.bib}
\bibliographystyle{ieeetr}

\end{document}