\documentclass[10pt]{article}      % use "amsart" instead of "article" for AMSLaTeX format
\usepackage{geometry}                       % See geometry.pdf to learn the layout options. There are lots.
\geometry{letterpaper}                          % ... or a4paper or a5paper or ... 
%\geometry{landscape}                       % Activate for rotated page geometry
%\usepackage[parfill]{parskip}          % Activate to begin paragraphs with an empty line rather than an indent
\usepackage{graphicx}               % Use pdf, png, jpg, or eps§ with pdflatex; use eps in DVI mode
                                % TeX will automatically convert eps --> pdf in pdflatex   
\usepackage{caption}
\usepackage{subcaption}
\captionsetup{justification=raggedright,singlelinecheck=false}

\usepackage{amsmath}
\usepackage{amsthm}
\usepackage{amsfonts}
\usepackage{verbatim}
\usepackage{bm}
\usepackage{multicol}
\usepackage[switch]{lineno}

\usepackage{tikz}
\usetikzlibrary{fit,positioning}

\newcommand{\cov}{\mathrm{cov}}

\usepackage{setspace}
\doublespacing

\begin{document}
\title{Learned Cumulant Generating Functions Allow Adaptation of Neural Network Statistics}
\author{Luke Rast}
\maketitle
\linenumbers


\section{Abstract}






\section{Introduction and Background}


The cumulant generating function has properties that place it at the heart of change detection and adaptation.




In order to make this paper tight, we must show that the learned CGF solves multiple issues related to adaptation / distribution change.
We already have
\begin{enumerate}
  \item Change point detection via rate function
  \item adaptation through training on tilted distribution
\end{enumerate}



While the term `rate function' can have a variety of meanings, through-out this paper, it will always refer to the large-deviations rate function, which we define in detail below.





\subsection{Substance Intro}


In this work, we use learned moment generating functions to track neural network statistics 


The \textit{moment generating function} (MGF) is given by
\begin{equation}
  M(\theta) = \mathbb{E}_x(e^{\theta x}), \label{def:MGF}
\end{equation}
the Laplace transform of the probability density function.
When this function exists, it gives an alternative represention of a probability distribution that is used frequently in probability and statistics [Ref-general MGF].
One important application of the MGF is for constructing families of \textit{exponentially tilted} probability distributions \cite{morris_natural_1982,morris_unifying_2009}.
Starting from a baseline distribution $h_0(x)$, with moment generating function $M_0(\theta)$, exponential tilting produces the family
\begin{equation}
  p_0(x | \theta) = e^{\theta^T x} \frac{h_0(x)}{M_0(\theta)}. \label{def:exponential_tilt}
\end{equation}
This is a natural exponential family of distributions, and the same process can be extended to produce any exponential family, by combining a set of statistics of the data, $T(x)$, and a base distribution of these statistics.
This is the path that we pursue in this paper: we will learn exponential families by learning both $T(x)$ and $M(\theta)$ and use these learned families to perform inference.

The logarithm of the MFG is called the \textit{cumulant generating function} (CGF):
\begin{equation}
  A(\theta) = \log \mathbb{E}_x[\exp(\theta x)]. \label{def:cgf}
\end{equation}
This function, if it exists, is guaranteed to be \textit{convex}.
Paralleling other applications of the Laplace transform [REF-physics uses] the CGF maintains many of the relationships that the MGF has to other quantities, through its Legendre transform [Ref - Legendre]:
\begin{equation}
  I(\mu) = \sup_{\theta}( \theta \mu - A(\theta) ). \label{eq:legendre_transform}
\end{equation}
We will take advantage of some of these well-known relationships in this paper.

First, the exponentially tilted distributions become reconizably exponential families
\begin{equation}
  p(T | \theta) = h(T) \exp( \theta^T T - A(\theta) ),
\end{equation}
and the CGF extends to produce cumulant generating functions for the whole family: $K(t | \theta) = A(\theta + t) - A(\theta)$.
What's more, because the CGF $A(\theta)$ is convex, the Legendre transform relationship will be a duality, and therefore we have the relationships:
\begin{eqnarray}
  \mu = \nabla_\theta A(\theta) & \theta = \nabla_\mu I(\mu). \label{eq:duality_relations}
\end{eqnarray}
The parameter $\mu$ is the derivative of the CGF, and therefore is equal to the average value of $T$ for data generated from the corresponding distribution $p(T|\theta)$.
Thus, $I(\mu)$ is a function of observation \textit{averages}.

This function appears in large deviations theory \cite{dembo2009large,touchette_basic_2012} as what is called the \textit{rate function} (or more accurately, it's negative).
\begin{equation}
  -I(\mu) = \lim_{N \to \infty} \frac{1}{N} \log \left( p \left(\frac{1}{N} \sum_{i=1}^N X_i \ge \mu \right) \right)
\end{equation}
In other words, $I(\mu)$ gives the asymptotic \textit{decay rate}, as we increase the number of samples, of the probability that the empirical mean deviates from the true mean by at least $\mu$.
\begin{equation}
  p \left( \frac{1}{N} \sum_{i=1}^N X_i \ge \mu \right) \approx \exp(-N I(\mu))
\end{equation}
This result is known as Cramer's theorem \cite{dembo2009large}.
In this work, we use this as a means of detecting changes in environmental statistics. 


\subsubsection{Large Deviations rewrite}
% written after I studied the large deviations theory

The theory of large deviations \cite{dembo2009large,touchette_basic_2012} describes the rate at which rare events become more and more unlikely as we collect more samples. 
So, for a series of probability measures, $p_n$, large deviations theory gives a \textit{rate function}, $I(x)$ that, loosely speaking, satisfies
\begin{equation}
  p_n(x \in \Gamma) \sim \exp \left( - n \min_{x \in \Gamma} I(x) \right) \quad \textrm{as} \quad n \to \infty
\end{equation}
for any measurable set $\Gamma$.
That is, the probability of outcomes in the set $\Gamma$ shrinks exponentially with $n$ at a rate that is dominated by the smallest value of $I(x)$ on the set $\Gamma$ (see \cite{dembo2009large} for more precise expressions).
In particular, if we consider the average $\mu_n = \frac{1}{n} \sum_{i=1}^n X_i$ of $n$ i.i.d. data points $X_i$ in $\mathbb{R}^n$, the large deviations theory provides a refinement of the law of large numbers and the central limit theorem, through Cramer's theorem \cite{dembo2009large} (for $\mathbb{R}^n$), which states that the rate function is given by
\begin{equation}
  I(\mu) = \sup_{\theta}( \theta \mu - A(\theta) ). \label{eq:legendre_transform}
\end{equation}
That is, the rate function is the Legendre transform of the cumulant generating function.







\subsection{Previous work}

Previous works on 

Previous works on exponential tilting
- finance
- ML

What is the connection to previous free energy based approaches?



\section{Results}

Learn cumulant generating function (CGF)
\begin{equation}
  A(\theta) = \log \mathbb{E}_x[\exp(\theta x)]
\end{equation}
for the activity of neural on a network, as a means to capture the statistics of these layers.


\subsection{Learned moment generating function}

The cumulant generating function is constrained to be a convex function of its argument, $\theta$.
We learn the cumulant generating function by training an input convex neural network (ICNN) \cite{amos_input_2017}, guaranteed to be convex, to reproduce the empirical cumulant generating function
\begin{equation}
  \hat A(\theta) = \log \sum_{x_i} \exp(\theta x),
\end{equation}
which is frequently used as a estimator for the CGF \cite{duffield_entropy_1995} [others]. 


Question: how many training samples do I need as a function of dimensionality to learn the CGF? This could be an argument for using the dimensionality reduction by the neural network features.


\subsection{Change point detection}
The cumulant generating function, $A(\theta)$ can be used to detect changes in the statistics of the network layers by taking advantage of tools from large deviations theory \cite{touchette_basic_2012} [others], which analyzes the decay behavior of the tails of distributions of sample averages.
The connection is made through Cramer's theorem [ref], which states that the large deviations \textit{rate function}
\begin{equation}
  I(x) = \lim_{n \to \infty} \frac{1}{n} \log \left( p \left( \sum_{i=1}^n X_i \ge n x \right) \right)
\end{equation}
is the negative of the Legendre transform of the cumulant generating function
\begin{equation}
  -I(x) = \sup_{\theta}( \theta x - A(\theta) ).
\end{equation}
In other words, we can use the Legendre transform to approximate cumulative distribution function of averages
\begin{equation}
  p \left( \frac{1}{n} \sum_{i=1}^n X_i \ge x \right) \approx \exp(-n I(x))
\end{equation}
Averaging enough data points will allow us to determine whether the network activity follows the same statistics as the training data using either element-wise test on each dimension, as in \cite{rabanser_failing_2019} or a joint test on the all dimensions simultaneously.


Steps:
\begin{enumerate}
  \item Compute Legendre transforms
  \item Empirical path-independence of Legendre transforms
  \item Use for change-point detection
  \item Compare sensitivity
\end{enumerate}



\subsection{Exponential tilting}
Given a learned cumulant generating function, $A(\theta)$ and an estimate of the density $h(T)$, we can use what is called `exponential tilting' \cite{morris_natural_1982,morris_unifying_2009} to construct a natural exponential family of distributions using this function as the base measure by:
\begin{equation}
  p(T | \theta) = h(T) \exp( \theta^T T - A(\theta)).
\end{equation}
This gives us parameterized family of distributions with parameter $\theta$, and cumulant generating functions $C(t) = A(\theta +t) - A(\theta)$.
The Fisher information matrix for this family is given by the Hessian of $A(\theta)$ 
\begin{equation}
  I(\theta) = \bm{H}_\theta A(\theta) 
\end{equation}

Steps / questions:
\begin{enumerate}
  \item How does exponential tilting act on the distribution?
  \item Does this capture the distribution given a particular value of the stimulus?
  \item What about if we construct a circle through the activity space?
  \item Use for Bayesian inference.
\end{enumerate}



\subsection{Adaptation Through Fine tuning}





\subsection{Target conditional CGF}





\subsection{Adaptation Through Activity Remapping}






\subsection{Measurement of Fisher Information}

Establish the method for measuring the Fisher information of data. In particular, with respect to changes in a known parameter, which can be trained.




\section{Discussion}

Learning an exponential family by learning sufficient statistic $T(x)$ and the base measure $h(x)$, here we use the features learned by a pre-trained neural network as sufficient statistics, and capture their distribution with learned moment generating function $A(\theta)$.

Future direction: learning $T(x)$ together with sufficient statistics.

Connection to VAEs \cite{kingma_auto-encoding_2013}: inner vs outer relaxation of inference.
VAEs learn to perform approximate inference with specified noise and latent variable distributions, while the approach here performs exact inference using learned noise distributions and their conjugates. 


\section{Methods}
\subsection{Network architecture}

We fit the empirical cumulant generating functions using input convex neural networks \cite{amos_input_2017}, which are guaranteed to learn convex functions.
We use a multi-layer perceptron with no skip connections, but with positive weight matrices initialized as in \cite{hoedt_principled_2023}, and guaranteed to be positive by a ReLU applied to the weights.
For non-linearities, we use a leaky version of a softplus, given by $sp_{\textrm{leaky}}(x) = \alpha x + (1 - \alpha) sp(x) + c_0$ where $sp(x)$ is a softplus function, $\alpha$ sets the leak slope for negative values, and $c_0$ sets the intercept. 
These values are set by hyper-parameter sweep (???). 
The leaky softplus non-linearity has appealing properties:
It is continuously differentiable to at least second order, giving smooth Jacobians and Hessians.
It retains benefits of leaky ReLUs in preventing dead units.
Finally, it allows negative values to pass through the network, which is particularly important in the ICNN case, where the network weights are all positive.




\subsection{Duality Computations}
Given a neural network representation of the cumulant generating function $A(\theta)$, we want to be able to transform between parameter and moment representations $\theta \leftrightarrow \mu$, and to evaluate both CGF and rate function values, $A(\theta)$ and $I(\mu)$.
Using the trained network, $A(\theta)$ is directly computed by evaluation, while the parameter to moment mapping $\mu(\theta)$ is given by its Jacobian (eq.~\ref{eq:duality_relations}).
Evaluation of $I(\mu)$ and $\mu(\theta)$ both come down to evaluating the Legendre transform (eq.~\ref{eq:legendre_transform}). 
We directly solve the optimization problem by gradient descent in $\theta$, with the resulting optimal input giving $\theta(\mu)$, while the optimal value is $I(\mu)$.
This works well enough for our purposes, but more sophisticated optimization and differential equations based approaches are available as well [REF???].



\bibliography{main.bib}
\bibliographystyle{ieeetr}

\end{document}